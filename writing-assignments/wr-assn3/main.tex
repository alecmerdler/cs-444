\documentclass[letterpaper,draftclsnofoot,10pt,onecolumn,titlepage]{IEEEtran}\usepackage[margin=0.75in]{geometry}
\renewcommand\thesection{\arabic{section}}
\renewcommand\thesubsection{\thesection.\arabic{subsection}}
\renewcommand\thesubsubsection{\thesubsection.\arabic{subsubsection}}
\renewcommand\thesectiondis{\arabic{section}}
\renewcommand\thesubsectiondis{\thesectiondis.\arabic{subsection}}
\renewcommand\thesubsubsectiondis{\thesubsectiondis.\arabic{subsubsection}}

\usepackage{graphicx}
\usepackage{url}
\usepackage{float}
\usepackage{color}
\usepackage{graphicx}
\usepackage{enumitem}
\usepackage[justification=centering]{caption}
\usepackage{listings}
\definecolor{backcolour}{rgb}{0.95,0.95,0.92}
\lstset{language=C++,
        basicstyle=\ttfamily,
        keywordstyle=\color{blue}\ttfamily,
        stringstyle=\color{red}\ttfamily,
        commentstyle=\color{green}\ttfamily,
        morecomment=[l][\color{magenta}]{\#},
        backgroundcolor=\color{backcolour}
}
\usepackage{geometry}
\graphicspath{{../img/}}
\usepackage{parskip}
\setlength{\parindent}{0pt}
\usepackage{hyperref}
\usepackage{geometry}
\linespread{1}
\setcounter{secnumdepth}{5}
\setcounter{secnumdepth}{5}

\title{Writing Assignment 3}

\author{
	Alec Merdler\\
	\and
	CS 444: Operating Systems II\\
	\and
	Spring 2017\\
}

\date{June 2, 2017}

\begin{document}
\begin{titlepage}
\clearpage\maketitle
\thispagestyle{empty}

\maketitle
\end{titlepage}

\section{Introduction}
An essential function of computers is memory management. This is the process of loading data from disk and assigning
it to a contiguous block of main memory, called a page. This reduces the number of disk reads the operating system 
needs to perform, which was a fundamentally slow operation (until the advent of solid state drives). Memory can 
also be allocated to running programs to be used in their processes, and then needs to be cleaned up and 
reallocated.

\section{Windows}
The core memory management system in Windows is called the Memory Manager. The system has two primary tasks: mapping 
the virtual address space of a process to physical memory, and paging some contents of memory to disk. The Memory 
Manager ensures that when a process reads from or writes to memory, it is referencing the correct address. Paging 
memory contents to disk occurs when a running process attempts to use more virtual memory than the machine
has in physical memory.

\subsection{Design}
The Windows Memory Manager is comprised of several executive system services and components. These components 
include executive system services for allocating and deallocating memory, managing virtual 
memory, resolving hardware memory management exceptions, the balance set manager, the process/stack swapper, 
the modifier page writer, the mapper page writer, the segment deference thread, and the zero page thread.

\subsection{Implementation}

\subsubsection{Pages}
The Windows operating system uses pages as the smallest unit of memory protection at the hardware level.
Two types of pages are defined: small and large. Large pages have a significant speed advantage when 
translating virtual memory locations to physical addresses because the first reference to any byte in a 
large page causes the hardware translation look aside buffer to have the necessary information in its 
cache to translate references of any other byte in the page. Small pages do not have this advantage,
therefore the core of the Windows operating system is mapped to large pages.

\subsubsection{Memory Pools}
The Windows Memory Manager creates and maintains both a paged and nonpaged memory pool. The nonpaged 
pool consists of virtual memory addresses that are guaranteed to reside in physical memory at the 
time of allocation. The paged pool consists of virtual memory that can be paged. For performance,
single processor systems have three paged pools, while multiprocessor systems have five.

\subsection{Comparison to Linux}


\section{FreeBSD}
The memory management system in FreeBSD is hierarchical in nature. First, there is the main primary memory system.
This is usually random access memory (RAM). Secondary memory is usually hard disk or solid state drives. In 
this model, all processes run in the virtual address space. References to the virtual address space are 
interpreted by the hardware to references in physical memory by a process called address translation.


\subsection{Design}
FreeBSD implements a private address space for each of its processes. This address space is divided into three 
segments: text, data, and stack. The text segment is read only and contains the machine instructions of 
the program. It is where the programs code is stored and pulled during execution. The data segment 
contains initialized and uninitialized data of the program. The stack segment contains the runtime stack 
of the program.


\subsection{Implementation}

\subsection{Memory Allocation}
The allocation of memory in FreeBSD is accomplished through two main C library routines: $malloc()$ and $free()$.
This is very similar to other UNIX based operating systems. Below is a simple example of how to use these 
two powerful functions to allocate 10 bytes to a string, then free the memory:

\begin{figure}[H]
\begin{lstlisting}[language=C++]
char *str;
str = (char *) malloc(10);
free(str);
\end{lstlisting}
\end{figure}

\subsection{Comparison to Linux}
While the design and implementation of memory management in FreeBSD and Linux are very similar, there are 
some notable differences. One such difference is the use of superpages, which ensure that contiguous pages 
are optimized for processor speed and the address translation cache. The following code snippet shows the 
simplicity of the page struct in Linux, found in \url{linux/mm_types.h}:

\begin{figure}[H]
\begin{lstlisting}[language=C++]
struct page {
	unsigned long 			flags;
	atomic_t			_count;
	atomic_t			_mapcount;
	unsigned long			private;
	struct address_space		*mapping;
	pgoff_t				index;
	struct list_head		lru;
	void				*virtual;
};
\end{lstlisting}
\end{figure}


\section{Conclusion}


remove me\cite{Windows}
\bibliographystyle{IEEEtran}
\bibliography{biblio}

\end{document}
