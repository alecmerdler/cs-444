\documentclass[letterpaper,draftclsnofoot,10pt,onecolumn,titlepage]{IEEEtran}\usepackage[margin=0.75in]{geometry}
\renewcommand\thesection{\arabic{section}}
\renewcommand\thesubsection{\thesection.\arabic{subsection}}
\renewcommand\thesubsubsection{\thesubsection.\arabic{subsubsection}}
\renewcommand\thesectiondis{\arabic{section}}
\renewcommand\thesubsectiondis{\thesectiondis.\arabic{subsection}}
\renewcommand\thesubsubsectiondis{\thesubsectiondis.\arabic{subsubsection}}

\usepackage{graphicx}
\usepackage{url}
\usepackage{float}
\usepackage{color}
\usepackage{graphicx}
\usepackage{enumitem}
\usepackage[justification=centering]{caption}
\usepackage{listings}
\definecolor{backcolour}{rgb}{0.95,0.95,0.92}
\lstset{language=C++,
        basicstyle=\ttfamily,
        keywordstyle=\color{blue}\ttfamily,
        stringstyle=\color{red}\ttfamily,
        commentstyle=\color{green}\ttfamily,
        morecomment=[l][\color{magenta}]{\#},
        backgroundcolor=\color{backcolour}
}
\usepackage{geometry}
\graphicspath{{../img/}}
\usepackage{parskip}
\setlength{\parindent}{0pt}
\usepackage{hyperref}
\usepackage{geometry}
\linespread{1}
\setcounter{secnumdepth}{5}
\setcounter{secnumdepth}{5}

\title{Writing Assignment 3}

\author{
	Alec Merdler\\
	\and
	CS 444: Operating Systems II\\
	\and
	Spring 2017\\
}

\date{June 2, 2017}

\begin{document}
\begin{titlepage}
\clearpage\maketitle
\thispagestyle{empty}

\maketitle
\end{titlepage}

\section{Introduction}
An essential function of computers is memory management. This is the process of loading data from disk and assigning
it to a contiguous block of main memory, called a page. This reduces the number of disk reads the operating system 
needs to perform, which was a fundamentally slow operation (until the advent of solid state drives). Memory can 
also be allocated to running programs to be used in their processes, and then needs to be cleaned up and 
reallocated.

\section{Windows}
The core memory management system in Windows is called the Memory Manager. The system has two primary tasks: mapping 
the virtual address space of a process to physical memory, and paging some contents of memory to disk. The Memory 
Manager ensures that when a process reads from or writes to memory, it is referencing the correct address. Paging 
memory contents to disk occurs when a running process attempts to use more virtual memory than the machine
has in physical memory.

\subsection{Components}
The Windows Memory Manager is comprised of several executive system services and components. These components 
include executive system services for allocating and deallocating memory, managing virtual 
memory, resolving hardware memory management exceptions, the balance set manager, the process/stack swapper, 
the modifier page writer, the mapper page writer, the segment deference thread, and the zero page thread.


\section{FreeBSD}


\section{Conclusion}


\begin{lstlisting}[language=C++]

\end{lstlisting}


remove me\cite{Windows}
\bibliographystyle{IEEEtran}
\bibliography{biblio}

\end{document}
