\documentclass[letterpaper,draftclsnofoot,10pt,onecolumn,titlepage]{IEEEtran}\usepackage[margin=0.75in]{geometry}
\renewcommand\thesection{\arabic{section}}
\renewcommand\thesubsection{\thesection.\arabic{subsection}}
\renewcommand\thesubsubsection{\thesubsection.\arabic{subsubsection}}
\renewcommand\thesectiondis{\arabic{section}}
\renewcommand\thesubsectiondis{\thesectiondis.\arabic{subsection}}
\renewcommand\thesubsubsectiondis{\thesubsectiondis.\arabic{subsubsection}}

\usepackage{graphicx}
\usepackage{url}
\usepackage{float}
\usepackage{color}
\usepackage{graphicx}
\usepackage[justification=centering]{caption}
\usepackage{listings}
\definecolor{backcolour}{rgb}{0.95,0.95,0.92}
\lstset{language=C++,
        basicstyle=\ttfamily,
        keywordstyle=\color{blue}\ttfamily,
        stringstyle=\color{red}\ttfamily,
        commentstyle=\color{green}\ttfamily,
        morecomment=[l][\color{magenta}]{\#},
        backgroundcolor=\color{backcolour}
}
\usepackage{geometry}
\graphicspath{{../img/}}
\usepackage{parskip}
\setlength{\parindent}{0pt}
\usepackage{hyperref}
\usepackage{geometry}
\linespread{1}
\setcounter{secnumdepth}{5}
\setcounter{secnumdepth}{5}

\title{Writing Assignment 1}

\author{
	Alec Merdler\\
	\and
	CS 444: Operating Systems II\\
	\and
	Spring 2017\\
}

\date{March 28, 2017}

\begin{document}
\begin{titlepage}
\clearpage\maketitle
\thispagestyle{empty}

\maketitle
\end{titlepage}

\section{Introduction}
    \subsection{Processes}
    The explicit purpose of computers is to execute sets of instructions, called programs. As computers have increased
    in power and speed, the complexity and number of these programs has also increased. A simple program could count
    to the number ten, printing out each number along the way, then stop. The resources needed for a computer to
    execute this program would be minimal (barring any fancy recursion), and has only one explicit process: to
    count numbers. However, a more complex example would be a program that receives user input, performs extensive
    calculations, updates a database, and returns a result to the user. This type of program would be more aptly
    classified as a computer application, in that it contains many different components and processes, versus just
    counting numbers. Many of these processes can happen at the same time, and there can be duplicate processes in
    order to divide the work needed to be done. We can therefore think of a computer process as a program that is
    being executed, from which applications are built.

    \subsection{Threads}
    Each computer process requires resources in order to run, including the CPU, RAM, and hard disk storage. The
    operating system allocates these resources using threads.

    \subsection{CPU Scheduling}
    Any operating system that claims to be a multiprocessing operating system must have a way of managing all of
    the running processes and allocating system resources to them. One of the most critical resources is the CPU's
    time. Modern CPU's have multiple physical cores, which can execute their own process. However, in order to run
    even more processes at the same time, the operating system can quickly switch between different processes, giving
    them time to use the CPU, and creating the appearance of running many programs simultaneously.


\section{Linux}
    \subsection{Processes}
    Each process (also known as a task) in Linux is represented by a task\_struct data structure, which is naturally
    large and complex. Outlined below are the main basic functional areas and fields \cite{linuxprocesses}:

    \begin{itemize}
        \item State - Can be one of following: running, waiting, stopped, zombie
        \item Scheduling Information - Data regarding process precedence in the system
        \item Identifiers - Simply a unique number for the process
        \item Inter-Process Communication - Standard *nix information for processes to share data
        \item Links - Pointers to parent, sibling, and child processes
        \item Timers - Data about when the process was created and how much CPU time it has consumed
        \item File System - Pointers to working directories and other files
        \item Virtual Memory - How much virtual memory assigned to the process
        \item Processor Specific Context - Information that is needed to restart the process in the same state it was in
    \end{itemize}

    The tasks vector contains pointers to every task\_struct. The default size of the task vector is 512 entries, and
    the currently running process can be found using the current pointer.
    
    \subsection{CPU Scheduling}
    The Linux CPU scheduling algorithm revolves around two key data structures: runqueues and priority arrays. 
    Runqueues are defined by a struct in \textbf{kernel/sched.c}, and contains fields like locks, number of running
    tasks, CPU load, pointer to current task, and others. Priority arrays allow round robin scheduling and finding 
    the highest priority task occur in constant time, which is a key part of the Linux scheduler \cite{linuxscheduler}.

\section{FreeBSD}
    \subsection{Processes}
    Processes in FreeBSD are similar to those in Linux in many ways. Each process has a unique process ID (PID). Each
    process has one owner and group whose permissions determine which files and devices the process can open. Processes
    also have a tree structure, with parent, sibling, and child processes \cite{freebsdfork}.
    \\
    New processes are created using the fork family of system calls. There are a few variations of the fork system call:
    fork creates a complete copy of the parent process; rfork creates a copy that shares a selected set of resources from
    its parent; vfork handles virtual memory differently than fork. Forking a process involves three steps: allocation and
    initialization of a new child process structure, duplicating the context of the parent process, and scheduling the 
    new child process to run. During the context duplication, the following portions are copied \cite{freebsdfork}:
    \begin{itemize}
    	\item Process group and session
	\item Signal state (ignored, caught, and blocked signal masks)
	\item The dg\_nice scheduling parameter 
	\item A reference to the parent credentials
	\item A reference to the set of open files by the parent
	\item A reference to the parent limits
    \end{itemize}
    
    \subsection{CPU Scheduling}

\section{Windows}
    \subsection{Processes}
    Processes in Windows are similar to those in Linux in many ways. Each process contains a virtual address space,
    executable code, open handles to system objects, a security context, a unique process ID, environment variables,
    a priority class, minimum and maximum working set sizes, and at least one thread of execution. Processes also
    have a tree structure, with parent, sibling, and child processes.

    Below is a code snippet showing how to create a child process in Windows using C++ \cite{windowsprocesses}:
\begin{lstlisting}[language=C++]
STARTUPINFO si;
PROCESS_INFORMATION pi;

ZeroMemory( &si, sizeof(si) );
si.cb = sizeof(si);
ZeroMemory( &pi, sizeof(pi) );

// Start the child process.
if( !CreateProcess( NULL,   // No module name (use command line)
    argv[1],        // Command line
    NULL,           // Process handle not inheritable
    NULL,           // Thread handle not inheritable
    FALSE,          // Set handle inheritance to FALSE
    0,              // No creation flags
    NULL,           // Use parent's environment block
    NULL,           // Use parent's starting directory
    &si,            // Pointer to STARTUPINFO structure
    &pi )           // Pointer to PROCESS_INFORMATION structure
)
{
    printf( "CreateProcess failed (%d).\n", GetLastError() );
    return;
}
\end{lstlisting}

    \subsection{Threads}
    Each Windows process is started with a single thread, but can create more threads which all share its virtual
    address space and system resources. Threads can be created by using the CreateThread function, which returns a
    handle to the thread.
    
    \subsection{CPU Scheduling}
    The Windows CPU scheduler uses the round robin method to assign time slices to all threads of the highest 
    priority. If a thread with higher priority becomes ready to run, the scheduler will stop executing the lower
    priority thread immediately and assign a full time slice to the new thread \cite{windowsprocesses}. This can 
    be contrasted with the Linux scheduler, which utilizes runqueues to fairly assign time slices to threads 
    while preventing starvation \cite{linuxscheduler}.
    
    
\section{Conclusion}
The implementation of processes, threads, and multitasking are quite similar among Windows, FreeBSD, and Linux 
operating systems. Where they differ is often due to legacy code or business restrictions, rather than an 
engineering difference of opinion. After countless iterations of these implementations, the fact that they 
have converged to such a common platform seems to indicate our engineering efforts have yielded the best 
solution to this complex problem.

\begin{thebibliography}{4}

\bibitem{linuxprocesses}
\textit{Chapter 4 Processes}
David A Rusling, 1999, URL: \url{http://www.tldp.org/LDP/tlk/kernel/processes.html}

\bibitem{freebsdfork}
\textit{FreeBSD Man Pages - Fork(2)}
The FreeBSD Project, 2017, URL: \url{https://www.freebsd.org/cgi/man.cgi?fork(2)}

\bibitem{windowsprocesses}
\textit{Design and Implementation of a CPU Bound Process Migration in Windows 7}
A. M. Khidhir, B. A. Mustafa and N. T. Saleh, 2012, URL: \url{http://ieeexplore.ieee.org/stamp/stamp.jsp?tp=&arnumber=6516381&isnumber=6516308}

\bibitem{linuxscheduler}
\textit{Understanding the Linux 2.6.8.1 CPU Scheduler}
Josh Aas, 2005, URL: \url{https://www.cs.unm.edu/~eschulte/classes/cs587/data/10.1.1.59.6385.pdf}

\end{thebibliography}

\end{document}
