\documentclass[letterpaper,draftclsnofoot,10pt,onecolumn,titlepage]{IEEEtran}\usepackage[margin=0.75in]{geometry}
\renewcommand\thesection{\arabic{section}}
\renewcommand\thesubsection{\thesection.\arabic{subsection}}
\renewcommand\thesubsubsection{\thesubsection.\arabic{subsubsection}}
\renewcommand\thesectiondis{\arabic{section}}
\renewcommand\thesubsectiondis{\thesectiondis.\arabic{subsection}}
\renewcommand\thesubsubsectiondis{\thesubsectiondis.\arabic{subsubsection}}

\usepackage{graphicx}
\usepackage{float}
\usepackage{graphicx}
\usepackage[justification=centering]{caption}
%\usepackage{enumitem}
\usepackage{listings}
\usepackage{geometry}
\graphicspath{{../img/}}
\usepackage{parskip}
\setlength{\parindent}{0pt}
\usepackage{hyperref}
\usepackage{geometry}
\linespread{1}
\setcounter{secnumdepth}{5}
\setcounter{secnumdepth}{5}

\title{Writing Assignment 1}

\author{
	Alec Merdler\\
	\and
	CS 444: Operating Systems II\\
	\and
	Spring 2017\\
}

\date{March 28, 2017}

\begin{document}
\begin{titlepage}
\clearpage\maketitle
\thispagestyle{empty}

\maketitle
\end{titlepage}

\section{Introduction}
    \subsection{Processes}
    The explicit purpose of computers is to execute sets of instructions, called programs. As computers have increased
    in power and speed, the complexity and number of these programs has also increased. A simple program could count
    to the number ten, printing out each number along the way, then stop. The resources needed for a computer to
    execute this program would be minimal (barring any fancy recursion), and has only one explicit process: to
    count numbers. However, a more complex example would be a program that receives user input, performs extensive
    calculations, updates a database, and returns a result to the user. This type of program would be more aptly
    classified as a computer application, in that it contains many different components and processes, versus just
    counting numbers. Many of these processes can happen at the same time, and there can be duplicate processes in
    order to divide the work needed to be done. We can therefore think of a computer process as a program that is
    being executed, from which applications are built.

    \subsection{Threads}
    Each computer process requires resources in order to run, including the CPU, RAM, and hard disk storage. The
    operating system allocates these resources using threads.

    \subsection{CPU Scheduling}
    Any operating system that claims to be a multiprocessing operating system must have a way of managing all of
    the running processes and allocating system resources to them. One of the most critical resources is the CPU's
    time. Modern CPU's have multiple physical cores, which can execute their own process. However, in order to run
    even more processes at the same time, the operating system can quickly switch between different processes, giving
    them time to use the CPU, and creating the appearance of running many programs simultaneously.


\section{Linux}


\section{FreeBSD}


\section{Windows}

\end{document}
