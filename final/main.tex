\documentclass[letterpaper,draftclsnofoot,10pt,onecolumn,titlepage]{IEEEtran}\usepackage[margin=0.75in]{geometry}
\renewcommand\thesection{\arabic{section}}
\renewcommand\thesubsection{\thesection.\arabic{subsection}}
\renewcommand\thesubsubsection{\thesubsection.\arabic{subsubsection}}
\renewcommand\thesectiondis{\arabic{section}}
\renewcommand\thesubsectiondis{\thesectiondis.\arabic{subsection}}
\renewcommand\thesubsubsectiondis{\thesubsectiondis.\arabic{subsubsection}}

\usepackage{graphicx}
\usepackage{url}
\usepackage{float}
\usepackage{color}
\usepackage{graphicx}
\usepackage{enumitem}
\usepackage[justification=centering]{caption}
\usepackage{listings}
\definecolor{backcolour}{rgb}{0.95,0.95,0.92}
\lstset{language=C++,
        basicstyle=\ttfamily,
        keywordstyle=\color{blue}\ttfamily,
        stringstyle=\color{red}\ttfamily,
        commentstyle=\color{green}\ttfamily,
        morecomment=[l][\color{magenta}]{\#},
        backgroundcolor=\color{backcolour}
}
\usepackage{geometry}
\graphicspath{{../img/}}
\usepackage{parskip}
\setlength{\parindent}{0pt}
\usepackage{hyperref}
\usepackage{geometry}
\linespread{1}
\setcounter{secnumdepth}{5}
\setcounter{secnumdepth}{5}

\title{Final Writing Assignment}

\author{
	Alec Merdler\\
	\and
	CS 444: Operating Systems II\\
	\and
	Spring 2017\\
}

\date{June 10, 2017}

\begin{document}
\begin{titlepage}
\clearpage\maketitle
\thispagestyle{empty}

\maketitle
\end{titlepage}

\section{Introduction}
TODO

% Assignment 1

\section{Processes and Threads}
    \subsection{Introduction}
        \subsubsection{Processes}
        The explicit purpose of computers is to execute sets of instructions, called programs. As computers have increased
        in power and speed, the complexity and number of these programs has also increased. A simple program could count
        to the number ten, printing out each number along the way, then stop. The resources needed for a computer to
        execute this program would be minimal (barring any fancy recursion), and has only one explicit process: to
        count numbers. However, a more complex example would be a program that receives user input, performs extensive
        calculations, updates a database, and returns a result to the user. This type of program would be more aptly
        classified as a computer application, in that it contains many different components and processes, versus just
        counting numbers. Many of these processes can happen at the same time, and there can be duplicate processes in
        order to divide the work needed to be done. We can therefore think of a computer process as a program that is
        being executed, from which applications are built.

        \subsubsection{Threads}
        Each computer process requires resources in order to run, including the CPU, RAM, and hard disk storage. The
        operating system allocates these resources using threads.

        \subsubsection{CPU Scheduling}
        Any operating system that claims to be a multiprocessing operating system must have a way of managing all of
        the running processes and allocating system resources to them. One of the most critical resources is the CPU's
        time. Modern CPU's have multiple physical cores, which can execute their own process. However, in order to run
        even more processes at the same time, the operating system can quickly switch between different processes, giving
        them time to use the CPU, and creating the appearance of running many programs simultaneously.

    \subsection{Linux}
        \subsubsection{Processes}
        Each process (also known as a task) in Linux is represented by a task\_struct data structure, which is naturally
        large and complex. Outlined below are the main basic functional areas and fields \cite{linuxprocesses}:

        \begin{itemize}
            \item State - Can be one of following: running, waiting, stopped, zombie
            \item Scheduling Information - Data regarding process precedence in the system
            \item Identifiers - Simply a unique number for the process
            \item Inter-Process Communication - Standard *nix information for processes to share data
            \item Links - Pointers to parent, sibling, and child processes
            \item Timers - Data about when the process was created and how much CPU time it has consumed
            \item File System - Pointers to working directories and other files
            \item Virtual Memory - How much virtual memory assigned to the process
            \item Processor Specific Context - Information that is needed to restart the process in the same state it was in
        \end{itemize}

        The tasks vector contains pointers to every task\_struct. The default size of the task vector is 512 entries, and
        the currently running process can be found using the current pointer.

        \subsubsection{CPU Scheduling}
        The Linux CPU scheduling algorithm revolves around two key data structures: runqueues and priority arrays. 
        Runqueues are defined by a struct in \textbf{kernel/sched.c}, and contains fields like locks, number of running
        tasks, CPU load, pointer to current task, and others. Priority arrays allow round robin scheduling and finding 
        the highest priority task occur in constant time, which is a key part of the Linux scheduler \cite{linuxscheduler}.

    \subsection{FreeBSD}
        \subsubsection{Processes}
        Processes in FreeBSD are similar to those in Linux in many ways. Each process has a unique process ID (PID). Each
        process has one owner and group whose permissions determine which files and devices the process can open. Processes
        also have a tree structure, with parent, sibling, and child processes \cite{freebsdfork}.
        \\
        New processes are created using the fork family of system calls. There are a few variations of the fork system 
        call: fork creates a complete copy of the parent process; rfork creates a copy that shares a selected set of
        resources from its parent; vfork handles virtual memory differently than fork. Forking a process involves 
        three steps: allocation and initialization of a new child process structure, duplicating the context of the 
        parent process, and scheduling the new child process to run. During the context duplication, the following 
        portions are copied \cite{freebsdfork}:
        
        \begin{itemize}
            \item Process group and session
            \item Signal state (ignored, caught, and blocked signal masks)
            \item The dg\_nice scheduling parameter 
            \item A reference to the parent credentials
            \item A reference to the set of open files by the parent
            \item A reference to the parent limits
        \end{itemize}

        \subsubsection{CPU Scheduling}
        FreeBSD uses runqueues and round robin scheduling similar to Linux. If a thread blocks on a CPU, the scheduler
        will move that thread to the sleepqueue to prevent it from affecting other threads.

    \subsection{Windows}
        \subsubsection{Processes}
        Processes in Windows are similar to those in Linux in many ways. Each process contains a virtual address space,
        executable code, open handles to system objects, a security context, a unique process ID, environment variables,
        a priority class, minimum and maximum working set sizes, and at least one thread of execution. Processes also
        have a tree structure, with parent, sibling, and child processes.

        Below is a code snippet showing how to create a child process in Windows using C++ \cite{windowsprocesses}:
\begin{lstlisting}[language=C++]
STARTUPINFO si;
PROCESS_INFORMATION pi;

ZeroMemory( &si, sizeof(si) );
si.cb = sizeof(si);
ZeroMemory( &pi, sizeof(pi) );

// Start the child process.
if( !CreateProcess( NULL,   // No module name (use command line)
    argv[1],        // Command line
    NULL,           // Process handle not inheritable
    NULL,           // Thread handle not inheritable
    FALSE,          // Set handle inheritance to FALSE
    0,              // No creation flags
    NULL,           // Use parent's environment block
    NULL,           // Use parent's starting directory
    &si,            // Pointer to STARTUPINFO structure
    &pi )           // Pointer to PROCESS_INFORMATION structure
)
{
    printf( "CreateProcess failed (%d).\n", GetLastError() );
    return;
}
\end{lstlisting}

        \subsubsection{Threads}
        Each Windows process is started with a single thread, but can create more threads which all share its virtual
        address space and system resources. Threads can be created by using the CreateThread function, which returns a
        handle to the thread.

        \subsubsection{CPU Scheduling}
        The Windows CPU scheduler uses the round robin method to assign time slices to all threads of the highest 
        priority. If a thread with higher priority becomes ready to run, the scheduler will stop executing the lower
        priority thread immediately and assign a full time slice to the new thread \cite{windowsprocesses}. This can 
        be contrasted with the Linux scheduler, which utilizes runqueues to fairly assign time slices to threads 
        while preventing starvation \cite{linuxscheduler}.
    
    \subsection{Conclusion}
    The implementation of processes, threads, and multitasking are quite similar among Windows, FreeBSD, and Linux 
    operating systems. Where they differ is often due to legacy code or business restrictions, rather than an 
    engineering difference of opinion. After countless iterations of these implementations, the fact that they 
    have converged to such a common platform seems to indicate our engineering efforts have yielded the best 
    solution to this complex problem.


% Assignment 2 

\section{I/O}
Computers are fundamentally useless without the ability to receive different inputs and emit unique output.
Whether the data comes from a hard drive, solid state drive, mouse, keyboard, or virtual reality headset,
the operating system needs to be able to translate and deliver the data to the various programs that need
it. Conversely, programs running on the operating system need a way to provide feedback back out to 
devices. The design and implementation of handling I/O differs between three major operating systems: Windows,
FreeBSD, and Linux. However, there are also many similarities amongst them as well.

    \subsection{Windows}
        \subsubsection{I/O Components}
        The Windows I/O System consists of many executive components that together manage devices.
        The Windows operating system uses a number of executive components that work together to manage input and 
        output devices connected to the computer \cite{windows}.

        \begin{description}
            \item[Security] 
            Uniform security and naming across devices in order to protect shareable resources.

            \item[Asynchronous]
            High performance asynchronous packet based I/O to allow for the creation and maintenance of scalable 
            applications.

            \item[Multiple Platforms]
            Services that allow drivers to be written in a high level language and easily ported between different 
            machine architectures.

            \item[Extensibility]
            Layering and extensibility to allow for the addition of drivers that tranparently modify the behavior 
            of other drivers or devices, without requiring any changes to the driver whole behavior or device is 
            modified.

            \item[Dynamic Loading]
            Dynamic unloading and loading of device drivers so that they can be loaded on demand and will not 
            consume system resources when not in use.

            \item[Plug and Play]
            Support for plug and play, where the system locates and installs drivers for newly detected hardware, 
            assigns them hardware resources they require, and also allows applications to discover and activate 
            device interfaces.

            \item[Power Management]
            Support for power management so that the system or individual devices can enter low power states.

            \item[File Systems]
            Support for multiple installable file systems, including FAT, the CD-ROM file sustem(CDFS), the 
            Universal Disk Format (UDF) file system, and the Windows file system (NTFS).

            \item[Driver Tools]
            Windows Management Instrumentation (WMI) support and diagnostics tools so that drivers can be managed 
            and monitored through WMI scripts and programs.
        \end{description}

        \subsubsection{I/O Processing}
        Windows provides a hardware abstraction layer (HAL) at the lowest level in order to encapsulate the different
        processor and interrupt signals and instead expose a uniform set of application programming interfaces (APIs)
        to device drivers. The next layer is the actual device drivers. This layer is followed by the I/O system
        layer, which includes the various executive components listed in the previous section. Most I/O
        operations in Windows are synchronous, which means the application thread will block while the device
        performs its data operations. Asynchronous I/O operations are enabled by specifying the FILE-FLAG-OVERLAPPED
        flag when calling the CreateFile function. This will allow the application to continue executing while the
        device performs the I/O operation \cite{windows}.

\begin{lstlisting}[language=C++]
HANDLE WINAPI CreateFile(
  _In_     LPCTSTR               lpFileName,
  _In_     DWORD                 dwDesiredAccess,
  _In_     DWORD                 dwShareMode,
  _In_opt_ LPSECURITY_ATTRIBUTES lpSecurityAttributes,
  _In_     DWORD                 dwCreationDisposition,
  _In_     DWORD                 dwFlagsAndAttributes,
  _In_opt_ HANDLE                hTemplateFile
); 
\end{lstlisting}

        Above is the implementation of the CreateFile function. This function is used to create or open files or
        other I/O devices. The Windows scheduler processes the file descriptors using a C-LOOK algorithm. Once
        the list contains requests, the drivers will iterate through the list from the lowest to highest sector
        \cite{Windows_CreateFile}.

    \subsection{FreeBSD}
        \subsubsection{I/O Components}
        There are multiple different types of file descriptors that are used for difference processes. These 
        file descriptors types and uses are as follows:

        \begin{itemize}
            \item  PIPE - pipe
            \item  FIFO - named pipe
            \item  SEM - POSIX semaphore
            \item  PTS - pseudo-teletype master device
            \item  DEV - device not reference by a vnode
            \item  MQUEUE - POSIX message queue
            \item  KQUEUE - event queue
            \item  CRYPTO - cryptographic hardware
            \item  SHM - POSIX shared memory
            \item  PROCDESC - process
            \item  VNODE - file or device
            \item  SOCKET - communication endpoint
        \end{itemize}

        These file descriptors are important because each data structure that is used to implement the 
        I/O scheduler has one of these types. Also, there are multiple different flags that go along 
        with these file descriptors that will tell the operating system whether to block or not or if 
        the process will be asynchronous or synchronous \cite{FreeBSDBook}.

        \subsubsection{I/O Processing}
        Below is the implementation of the standard disk scheduler in FreeBSD. A significant difference between 
        this scheduler and the Linux scheduler is the ability to delete, load, and unload. Delete is used to free
        resources when a scheduler is not in use. Load/unload are used can hook into the respective scheduler events
        to perform specific operations. Every time init is called, it will allocate a struct containing device 
        information, which is referenced using a file pointer \cite{FreeBSDBook}.

\begin{lstlisting}[language=C++]
struct _disk_sched_interface {
       struct _disk_sched_interface *next;
       char *name;     /* symbolic name */
       int refcount;   /* users of this scheduler */

       void (*disksort)(struct buf_queue_head *head, struct buf *bp);
       void (*remove)(struct buf_queue_head *head, struct buf *bp);
       struct buf *(*get_first)(struct buf_queue_head *head);
       void (*init)(struct buf_queue_head *head);
       void (*delete)(struct buf_queue_head *head);
       void (*load)(void);     /* load scheduler */
       void (*unload)(void);   /* unload scheduler */
};
\end{lstlisting}\cite{freebsd}

    \subsection{Conclusion}
    The design and implementation of I/O in FreeBSD is very similar to Linux, in that they both use C-LOOK algorithm.
    All three operating systems provide file descriptors to specify asynchronous data operations, but only FreeBSD
    requires the type of descriptor. Windows has several layers of abstraction, including the hardware 
    abstraction layer (HAL) to provide a uniform interface for all device drivers across different hardware 
    architectures. This is similar to the Linux kernel itself, which provides a uniform set of APIs across 
    countless devices. FreeBSD also has a form of HAL.


% Assignment 3

\section{Memory Management}
An essential function of computers is memory management. This is the process of loading data from disk and assigning
it to a contiguous block of main memory, called a page. This reduces the number of disk reads the operating system 
needs to perform, which was a fundamentally slow operation (until the advent of solid state drives). Memory can 
also be allocated to running programs to be used in their processes, and then needs to be cleaned up and 
reallocated.

    \subsection{Windows}
    The core memory management system in Windows is called the Memory Manager. The system has two primary tasks: mapping 
    the virtual address space of a process to physical memory, and paging some contents of memory to disk. The Memory 
    Manager ensures that when a process reads from or writes to memory, it is referencing the correct address. Paging 
    memory contents to disk occurs when a running process attempts to use more virtual memory than the machine
    has in physical memory \cite{windows}.

        \subsubsection{Design}
        The Windows Memory Manager is comprised of several executive system services and components. These components 
        include executive system services for allocating and deallocating memory, managing virtual 
        memory, resolving hardware memory management exceptions, the balance set manager, the process/stack swapper, 
        the modifier page writer, the mapper page writer, the segment deference thread, and the zero page thread
        \cite{windows}.

        \subsubsection{Implementation}
            \paragraph{Pages}
            The Windows operating system uses pages as the smallest unit of memory protection at the hardware level.
            Two types of pages are defined: small and large. Large pages have a significant speed advantage when 
            translating virtual memory locations to physical addresses because the first reference to any byte in a 
            large page causes the hardware translation look aside buffer to have the necessary information in its 
            cache to translate references of any other byte in the page. Small pages do not have this advantage,
            therefore the core of the Windows operating system is mapped to large pages \cite{windows}.

            \paragraph{Memory Pools}
            The Windows Memory Manager creates and maintains both a paged and nonpaged memory pool. The nonpaged 
            pool consists of virtual memory addresses that are guaranteed to reside in physical memory at the 
            time of allocation. The paged pool consists of virtual memory that can be paged. For performance,
            single processor systems have three paged pools, while multiprocessor systems have five \cite{windows}.

        \subsubsection{Comparison to Linux}
        One of the major differences in memory management between Linux and Windows is the way the two 
        operating systems handle shared memory. In Windows, shared memory is controlled by the Memory Manager
        using an abstraction called section objects. Section objects connected to either an open file or a 
        section of committed memory, and can be used by many processes to pretend that the page is actually
        in physical memory \cite{windows}. In Linux, shared memory are just virtual memory pages, created 
        using the $shmget()$ function \cite{linux}.

    \subsection{FreeBSD}
    The memory management system in FreeBSD is hierarchical in nature. First, there is the main primary memory system.
    This is usually random access memory (RAM). Secondary memory is usually hard disk or solid state drives. In 
    this model, all processes run in the virtual address space. References to the virtual address space are 
    interpreted by the hardware to references in physical memory by a process called address translation \cite{freebsd}.

        \subsubsection{Design}
        FreeBSD implements a private address space for each of its processes. This address space is divided into three 
        segments: text, data, and stack. The text segment is read only and contains the machine instructions of 
        the program. It is where the programs code is stored and pulled during execution. The data segment 
        contains initialized and uninitialized data of the program. The stack segment contains the runtime stack 
        of the program \cite{freebsd}.

        \subsubsection{Implementation}

            \paragraph{Memory Allocation}
            The allocation of memory in FreeBSD is accomplished through two main C library routines: $malloc()$ and
            $free()$. This is very similar to other UNIX based operating systems. Below is a simple example of how 
            to use these two powerful functions to allocate 10 bytes to a string, then free the memory:

\begin{figure}[H]
\begin{lstlisting}[language=C++]
char *str;
str = (char *) malloc(10);
free(str);
\end{lstlisting}
\end{figure}

            \paragraph{Shared Memory}
            It is often necessary for processes to communicate with each other. This interprocess communication can 
            easily be achieved using system calls, but these are often slow and require additional memory to execute.
            Instead, by allocating a shared memory address, any process can mutate and view changes to the memory 
            referenced by that address. This strategy runs the risk of one process corrupting the memory and affecting
            all other processes using that memory. In FreeBSD, this is avoided by using semaphores, which themselves 
            are located in shared memory to reduce the overhead of the necessary system calls \cite{freebsd}.

        \subsubsection{Comparison to Linux}
        While the design and implementation of memory management in FreeBSD and Linux are very similar, there are 
        some notable differences. One such difference is the use of superpages, which ensure that contiguous pages 
        are optimized for processor speed and the address translation cache. The following code snippet shows the 
        simplicity of the page struct in Linux, found in \url{linux/mm_types.h} \cite{linux}:

\begin{figure}[H]
\begin{lstlisting}[language=C++]
struct page {
	unsigned long 			flags;
	atomic_t			_count;
	atomic_t			_mapcount;
	unsigned long			private;
	struct address_space		*mapping;
	pgoff_t				index;
	struct list_head		lru;
	void				*virtual;
};
\end{lstlisting}
\end{figure}

    \subsection{Conclusion}
    Effective memory management is crucial to the performance of computer applications. Even programs 
    running on the latest high end hardware can underperform if the operating system is not taking 
    advantage of well established techniques, such as paging, address translation, and shared memory.
    The three major operating systems outlined here (Windows, FreeBSD, and Linux) have similar high level
    approaches to solving these problems, but the internal data structures and implementation of various
    system calls are what set them apart. Additionally, the handling of shared memory address space differs
    between them.


\section{Conclusion}
TODO


\bibliographystyle{IEEEtran}
\bibliography{biblio}

\end{document}
